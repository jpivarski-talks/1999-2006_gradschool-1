\documentclass[12pt]{article}
\usepackage{amsmath}

%\usepackage[pdftex]{graphicx}
\usepackage{graphicx}

\oddsidemargin  -0.5 cm
\evensidemargin 0.0 cm
\textwidth      6.5in
\headheight     0.0in
\topmargin      -1 cm
\textheight=9.0in

\title{Di-lepton Width of Upsilon(1S, 2S, 3S)}
\author{Jim Pivarski}

\begin{document}

\newcommand{\subs}[1]{{\mbox{\scriptsize #1}}}

\newcommand{\dxy}{$d_{XY}$}
\newcommand{\dz}{$d_Z$}
\newcommand{\pone}{$|\vec{p}_1|$}
\newcommand{\ptwo}{$|\vec{p}_2|$}
\newcommand{\eone}{$E_1$}
\newcommand{\etwo}{$E_2$}
\newcommand{\eisr}{$E_{\mbox{\scriptsize ISR}}$}
\newcommand{\visen}{$E_{\mbox{\scriptsize vis}}$}
\newcommand{\hotvisen}{$E_{\mbox{\scriptsize vis}}^{\mbox{\scriptsize hot}}$}
\newcommand{\lfourdec}{L4$_{\mbox{\scriptsize dec}}$}
\newcommand{\pdotp}{$|\vec{p}_1\cdot\vec{p}_2|$}
\newcommand{\ebeam}{$E_{\mbox{\scriptsize beam}}$}
\newcommand{\ecom}{$E_{\mbox{\scriptsize COM}}$}
\newcommand{\pz}{$\sum p_z$}

\newcommand{\scosmic}{$s_{\mbox{\scriptsize cosmic}}$}

\setlength{\baselineskip}{0.9 cm}

\setcounter{page}{1}
\pagenumbering{roman}
\pagestyle{plain}

\maketitle

\section*{Note for Paper Committee}

The $\Upsilon(1S)$, $\Upsilon(2S)$, and $\Upsilon(3S)$ di-lepton
widths ($\Gamma_{ee}$) are being calculated on the lattice by
Christine Davies's group at the University of Glasgow, likely to be
presented at HEP2005 Europhysics.  (That's why I'm aiming for the
same conference.)  My goal is to provide a precise experimental
measurement for comparison.  Currently, the experimental precisions on
$\Gamma_{ee}$ are 2.2\% (1S), 4.2\% (2S), and 9.4\% (3S): the expected
theoretical accuracy is 2--5\%.  My precisions are likely to be about
2\% for each resonance, so my measurements will confirm or refute
lattice predictions on the $\Upsilon(2S)$ and $\Upsilon(3S)$.  In this
paper, you will find most of the argument for that precision.  (A
checklist of what has been done and what will be done by the time of
the conference will follow.)

The experimental method is to fit lineshape scans for a total
$\Upsilon$ cross-section (integrated over beam energy).  From this
$\Gamma_{ee}$ can be derived:
\begin{equation}
  \Gamma_{ee} = \frac{{M_\Upsilon}^2}{6\pi^2} \, \int \sigma(e^+e^-
  \to \Upsilon) \, dE \mbox{.}
\end{equation}
(See the big PDG, page 837 (orange 2004), or Peskin and Schroeder,
page 151.)  Unlike most partial width measurements, the coupling of
$\Upsilon$ to $e^+e^-$ which is used is in the production
cross-section, not the decay.  The total cross-section will be
determined by counting all $\Upsilon$ decays except for $\Upsilon \to
e^+e^-$, $\mu^+\mu^-$, and cascades that end with a lower $\Upsilon$
resonance decaying to $e^+e^-$ or $\mu^+\mu^-$, and then adding these
back in using their known branching fractions.  (The easiest
anti-bhabha cut excludes anything with a high-momentum track,
such as these signal modes.)  It is customary to quote the total
{\it hadronic} cross-section, in which $\Upsilon \to e^+e^-$,
$\mu^+\mu^-$, and $\tau^+\tau^-$ are all left out, and only the
missing cascades are added back in.  This is because
$\Gamma_{ee}\Gamma_\subs{had}/\Gamma_\subs{tot}$ can be quoted with
higher precision and can avoid the assumption of lepton universality.
I will do both.

Seeing that my ultimate goal is to have accurate fits, my goal for
this paper is to precisely measure the hadronic cross-section of {\it
every} run in the $\Upsilon$ energy region.  This means that I need to
know the number of $\Upsilon$s very well (but not the number of
continuum events; I will be effectively subtracting them off with the
lineshape fits), the $e^+e^- \to \gamma\gamma$ (``gamgam'')
luminosity, and how these two might be affected by backgrounds and
instrumental glitches that can have a different dependence than
integrated luminosity.

In this paper, I have completed arguments for the following:
\begin{itemize}

  \item I know the $\Upsilon$ hadronic efficiency very well: 98.7
    $\pm$ 1.1\%, 96.7 $\pm$ 1.3\%, and 97.0 $\pm$ 1.3\% for the three
    resonances.

  \item I've searched for (and sometimes found) pathological runs by
    several general criteria, and can guarantee that the remaining
    runs are free of these issues.

  \item With my current set of gamgam cuts, there are no luminosity
    trigger issues.  I can't think of anything else that would vary
    run-by-run.

\end{itemize}

While you read this (and we have meetings, and you ask me questions
about it, and all that), I will continue to study the following
issues:
\begin{itemize}

  \item Fitting lineshape scans with Karl's fit function (includes
    Breit-Wigner $\otimes$ radiative corrections $\otimes$ beam energy
    spread Gaussian, where $\otimes$ is a convolution, with continuum
    interference in the Breit-Wigner).  I will vary all the lineshape
    distortions by their uncertainties and get more systematic errors
    for $\Gamma_{ee}$.  (Yes, I did this before, but with new hadron
    cuts and gamgam cuts, I want to be sure that the results don't
    change.)  Also, I want to check the beam energy calibration by
    artificially inserting miscalibrations, to see how much variation
    in $\Gamma_{ee}$ is possible before the fit $\chi^2$ becomes
    unbelievable.  I've never done this with real data or a real fit
    function.

  \item Translating gamgam counts into a luminosity measurement.  This
    will set the overall scale for $\Gamma_{ee}$ as a multiplicative
    factor that I can apply after all fitting.  Note that this doesn't
    need to be a gamgam luminosity measurement (in which I'd have to
    be sure I understand conversion probability and the effect of
    individual CC crystals biasing shower $\theta$ measurements).  I
    can simply compare my number of gamgams to a bhabha/mupair
    luminosity measurement (in which track-finding efficiency becomes
    the major issue).

\end{itemize}
Nothing in these last two studies will change what has been
established in this paper.

\end{document}
