\section{Scan and Fitting Methods}

Each resonance was scanned about ten times: approximately once a week,
CESR's beam energy was varied to independently scan the $\Upsilon$
lineshape.  This is because the measurement of beam energy can't be
guaranteed to its quoted precision from one week to the next, since
the NMR probe that measures it can be displaced during machine
studies, and a fit to lineshape data with unknown energy measurement
errors would be meaningless.  So each of these individual scans is
complete: it can be used to measure $\Gamma_{ee}$ independently.

Fitting each scan independently and then averaging $\Gamma_{ee}$,
however, would not be an optimal use of data.  Some scans don't have a
continuum point (cross-section measurement 20 MeV below the $\Upsilon$
mass), and only one scan at each resonance has a dedicated high-energy
point (20--50 MeV above the $\Upsilon$ mass).  Being far from the
resonance, these measurements aren't sensitive to shifts in beam
energy calibration on the order of an MeV.  (Shifts in apparent beam
energy between individual scans are $\lesssim$1 MeV.)  To optimally
use all the data, each resonance is fitted once with all scans
included, as well as a high-statistics continuum point and a
high-energy point.  Each scan's beam energy calibration is represented
as a floating parameter in the fit.

Each of these three fits allows the continuum background to float
freely: I am not restricting it to any functional form over the
hundreds of MeV from one resonance to the next.  Under a resonance
lineshape (a span of tens of MeV), I will have to assume a given
functional form, and the data are insufficient to let it float in the
fit.






\section{The Scan Periods}

\section{Fitting Continuum Points}

\section{Fitting Each Resonance}

\section{Simulating Beam Energy Calibration Shifts}





