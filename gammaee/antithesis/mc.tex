\section{Upsilon Monte Carlo}

An implementation of EvtGen was used to simulate $\Upsilon$ decays,
with each of the following generated separately: $\Upsilon \to
e^+e^-$, $\mu^+\mu^-$, $\tau^+\tau^-$, $ggg$, $gg\gamma$, $q\bar{q}$,
and $\Upsilon(2,3S) \to$ all cascade decays.

The three modes with free quarks and gluons are hadronized by JetSet
7.4 before being passed to the detector simulation.  This
hadronization step is an approximation, and that approximation must be
tested.  Another assumption made by the Monte Carlo is that the above
is an exhaustive list of $\Upsilon$ decays (where ``all cascade
decays'' includes only modes listed in the PDG [\ref{cite:pdg}]).  The
study described in the next Chapter will check these assumptions for
$\Upsilon(1S)$, though I will need to assume that cascade decays of
the $\Upsilon(2S)$ and $\Upsilon(3S)$ are well-described by the Monte
Carlo.

The detector simulation and reconstruction were executed using the
same version of code as in the database and unfiltered datasets.  This
is the ``MC code release'' in Table \ref{datasets:unfiltered}
(reconstruction code is guaranteed to be the same as that in the
corresponding ``data code release'').  Where data at one resonance
were processed using two different versions of the code
($\Upsilon(1S)$ and $\Upsilon(2S)$), the same Monte Carlo events were
passed through the different code versions, as a stringent test of
code reproducibility.

This Monte Carlo sample does suffer from one known bug: the
bunch-finding simulation, which is supposed to reproduce CLEO's
identification of which storage ring bunch actually collided to
produce the event, sometimes (rarely) returns too many tested bunches
and an incorrectly identified bunch number.  These failures can be
recognized, and all Monte Carlo tests (except the test for sensitivity
to this bug) rejected bad bunch-finding events.

Extra $\Upsilon$ Monte Carlo was generated for two decay modes.  The
first of these is $\Upsilon(2S) \to \pi^+\pi^- \Upsilon(1S)$, the
``cascade'' decay mode to be studied extensively in the next Chapter.
The other is $\Upsilon(2S,3S) \to \gamma \chi_b(1P,2P) \to \gamma
\gamma \gamma$.  This is a possible background to the gamgam event
type which is sought in Chapter \ref{chp:gamgambkgnds}.  No such
events are seen in real data.

All control files used to generate and process Monte Carlo are listed
in Appendix \ref{chp:appendixmc}.

\section{Gamgam Monte Carlo}

\section{Bhabha Monte Carlo}

Whatever is used for absolute luminosity\ldots


